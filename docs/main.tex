\documentclass[12pt, twoside]{article}
\usepackage{geometry}
\usepackage{caption}
\usepackage{subcaption}
\geometry{a4paper}

\usepackage[mono=false]{libertine} 

\usepackage{amsfonts}
\usepackage{amsmath}
\usepackage{amssymb}
\usepackage{amsthm}
\usepackage{etoc}

\usepackage[toc,title,titletoc,header]{appendix}
\usepackage{minitoc}
\usepackage{cancel}
\usepackage{imakeidx} % before hyperref
\usepackage{hyperref}
\usepackage{graphicx}
\usepackage{wrapfig}
\graphicspath{ {img/} }
\usepackage{verbatim}
\usepackage{setspace}
\usepackage{epigraph}
\usepackage[italian]{babel}
\usepackage[T1]{fontenc}
\usepackage[utf8]{inputenc}
\usepackage{a4}
\usepackage{csquotes}
\usepackage{url}
\usepackage{lettrine}
\usepackage{graphicx}
\usepackage{lscape}
\usepackage{epsfig}
\usepackage{epstopdf}
\usepackage{eurosym}
\usepackage{booktabs}
\usepackage{enumitem}
\usepackage{multirow}
\usepackage{multicol}
\usepackage{lscape}

\newenvironment{footcitedquote}[1][]
  {% \begin{footcitedquote}[.]
   \let\mkcitation\footnote% Citations will be footnotes
   \def\optarg{#1}% Capture optional argument
   % Set up start of displayquote environment
   \edef\envstart{\noexpand\begin{displayquote}\if$\optarg$\else[\noexpand\optarg]\fi}%
   \envstart}
  {\end{displayquote}}% \end{footcitedquote}

\title{Zillow’s Home Value Prediction}
\author{Matteo Spanio}
\date{Settembre 2022}



\begin{document}
    \begin{titlepage}
    \begin{center}
        \includegraphics[width=0.6\textwidth]{img/long_logo.jpg}
        
        \vspace{2.5cm}
        \Huge
        \textbf{Data and Web mining}
 
        \vspace{0.5cm}
        \LARGE
 
        \vspace{1cm}
 
        \textbf{Prof. Claudio Lucchese}
 
        \vfill
 
        Zillow’s Home Value Prediction
        
        \vfill

        \textbf{Matteo Spanio}\\
        {Matricola: 877485}
 

        \vspace{0.8cm}
        \Large
        Settembre 2022
 
    \end{center}
\end{titlepage}

    
    \onehalfspacing

    %\tableofcontents

    \section{The dataset}
    
    Di fronte al dataset fornito da Zillow la prima problematica da affrontare è scegliere come aggregare i dati: i file contenenti le transazioni con $logerror$ e data infatti trattano più volte la vendita della stessa proprietà. Unire i dati sulla chiave $parcelid$ creerebbe diverse righe duplicate tranne che per pochi valori, mentre includere questi duplicati permetterebbe di monitorare, a parità di condizioni della proprietà, cambiamenti di $logerror$ nel tempo, un dato che si intuisce essere importante dal format del file .csv per la consegna delle predizioni.

    \section{Features}
    ciao

    \section{Model selection}
    ciao

    \section{About outliers}

    \section{Conclusions}
    ciao

\end{document}